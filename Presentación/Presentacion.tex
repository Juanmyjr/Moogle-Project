\documentclass [12pt]{beamer}
\usepackage[utf8]{inputenc}
\usepackage{enumitem}
\usepackage{amsmath, amssymb, amsfonts}
\usepackage{graphicx}
\setbeamertemplate{caption}[numbered]

\usetheme {Frankfurt}
\institute[MATCOM]{\large "Facultad de Matemática y Computación"}
\author[Juan Miguel Maestre Rodriguez]{\textbf{Estudiante:}\textit{ Juan Miguel Maestre Rodriguez \\ C-121}}


\date{\today}
\title{\textbf{Moogle!}}


\begin{document}

\begin{frame}
\maketitle   

\begin{center}
   \includegraphics[height = 3 cm]{matcom.jpg}
\end{center}

\thispagestyle{empty}

\end{frame}

\begin{frame}{Clases}
\textbf{El programa consta de 8 clases} :
\begin{itemize}
\item DataBase
\item Normalize
\item TF-IDF
\item Snippet
\item Score
\item Moogle
\item SearchItem
\item SearchResult
\end{itemize}
\end{frame}

\begin{frame}{Clase DataBase}
Esta es la clase que se utiliza para crear y procesar la base de datos con la cual trabajará el programa.
La misma contiene solo un método llamado "Loading"
\end{frame}

\begin{frame}{Clase TF-IDF}
Esta clase tiene como objetivo medir cuán relevante es una palabra para un
documento en una colección. La misma contiene dos metodos: 
\begin{itemize}
    \item \textbf{TF}
    \item \textbf{IDF}
\end{itemize}
\end{frame}

\begin{frame}{Clase TF-IDF}
    $\textbf{TF} = np / nd$

    \begin{itemize}
        \item np : cantidad de veces que se repite una palabra en un documento 
        \item nd : cantidad de palabras que contiene el documento
    \end{itemize}
    


    $\textbf{IDF} = \log (t/n)$
    \begin{itemize}
        \item n : es la cantidad de documentos en los que aparece la palabra
        \item t : es la cantidad total de documentos
    \end{itemize}

\end{frame}

\begin{frame}{Clase Normalize}
Esta clase tiene como objetivo procesar todos los textos del proyecto para que en caso de que exista algún caracter 
que no sea reconocido por el programa procesarlo de forma adecuada y que no ocurran errores en el funcionamientos del mismo. 
Esta contiene un solo método llamado "Normalize"
\end{frame}

\begin{frame}{Clase Snippet}
Esta clase tiene como objetivo mostrar un fragmento de hasta 100 caracteres de los documentos que mayor coincidencia tengan con la 
búsqueda realizada por el programa. La misma contiene solamente un método llamado "ShowWords" 
\end{frame}

\begin{frame}{Clase Score}
Esta clase tiene como objetivo evaluar el nivel de relevancia que tiene cada documento con la búsqueda realizada por el programa.
La misma contiene solamente un método llamado "Ranking"
\end{frame}

\begin{frame}{Clase Moogle}
Esta es la clase principal del proyecto puesto que en ella es donde mayormente se utilizan los métodos creados en las clases anteriores.
Cabe mencionar que esta clase junto a SearchItem y a SearchResult existían antes de ser implementado el motor de búsqueda.
\end{frame}

\begin{frame}
    
\begin{center}
        \huge{FIN!}   
 \end{center}

\end{frame}
    
\end{document}